\documentclass{article}
\newcommand{\assgnnum}{2}
\newcommand{\duedate}{October 25, 10:30 AM}

\usepackage{amsmath}
%\usepackage{fullpage}
\usepackage{amssymb}
%\usepackage{bbm}
\usepackage{fancyhdr}
%\usepackage{paralist}
\usepackage{graphicx}
\usepackage{caption}
\usepackage{subcaption}
\usepackage[pdftex,colorlinks=true, urlcolor = blue]{hyperref}
\usepackage{listings}
\usepackage{color}
\usepackage{xcolor}

\definecolor{thegreen}{rgb}{0,.5,0}
\definecolor{comment-green}{rgb}{0,.3,0}
\definecolor{theblue}{rgb}{0,0,.8}
\definecolor{light-gray}{gray}{0.98}
\definecolor{comment-color}{rgb}{0,0,.8}
\definecolor{string-color}{rgb}{0,.75,0}
\definecolor{border-blue}{rgb}{0,0,.6}

\lstset{% use our version of highlighting
  language=python,      % using python
  keywordstyle={\color{teal}\bfseries},          % keywords
  commentstyle=\color{comment-color},         % comments
  stringstyle=\color{string-color},                   %strings
}

\lstset{
  basicstyle={\ttfamily\normalsize},  % use font and smaller size
  basewidth={0.5em,0.5em},
  showstringspaces=false,                   % do not emphasize spaces in strings
  tabsize=2,                                % number of spaces of a TAB
  aboveskip={0\baselineskip},               % a bit of space above
  columns=fixed,                            % nice spacing
}


\oddsidemargin 0in \evensidemargin 0in
\topmargin -0.5in \headheight 0.25in \headsep 0.25in
\textwidth 6.5in \textheight 9in
\parskip 6pt \parindent 0in \footskip 20pt

% set the header up
\fancyhead{}
\fancyhead[L]{CME193: Homework \assgnnum}
\fancyhead[R]{Due: \duedate}
%%%%%%%%%%%%%%%%%%%%%%%%%%
\renewcommand\headrulewidth{0.4pt}
\setlength\headheight{15pt}

\newcommand{\p}{\ensuremath{\mathbf{P}}}
\renewcommand{\Pr}[1]{\ensuremath{\p \left \{ #1 \right \}}}
\newcommand{\nti}{\ensuremath{n \to \infty}}
\newcommand{\I}{\ensuremath{\operatorname{I}}}
\newcommand{\One}[1]{\ensuremath{\mathbbm{1}_{\left \{ #1 \right \}}}}
\newcommand{\E}{\ensuremath{\mathbf{E}}}
\newcommand{\Ex}[2][]{\ensuremath{\E_{#1} \left[ #2 \right]}}
\newcommand{\var}{\ensuremath{\operatorname{Var}}}
\newcommand{\cov}{\ensuremath{\operatorname{Cov}}}
\newcommand{\F}{\ensuremath{\mathcal{F}}}
\newcommand{\R}{\ensuremath{\mathbb{R}}}
\newcommand{\C}{\ensuremath{\mathbb{C}}}
\newcommand{\NormRV}[2]{\ensuremath{\operatorname{N}\left(#1, #2\right)}}
\newcommand{\BetaRV}[2]{\ensuremath{\operatorname{Beta}\left(#1, #2\right)}}
\newcommand{\argmax}{\operatornamewithlimits{argmax}}
\newcommand{\x}{\mathbf{x}}
\newcommand{\A}{\mathbf{A}}
\newcommand{\bb}{\mathbf{b}}

\newcounter{points}
\setcounter{points}{0}
\newcounter{bonuspoints}
\setcounter{bonuspoints}{0}

\newcommand\setpoints[1]{\addtocounter{points}{#1}(#1 points)}
\newcommand\setpoint{\addtocounter{points}{1}(1 point)}
\newcommand\setbonuspoints[1]{\addtocounter{bonuspoints}{#1}(#1 bonus points)}
\newcommand\setbonuspoint{\addtocounter{bonuspoints}{1}(1 bonus point)}

\newcommand\printpoints{Total number of points: \value{\thepoints}}

\newcommand{\eqD}{\ensuremath{\overset{\mathcal{D}}{=}}}

\setlength{\parindent}{0in}

\begin{document}

\pagestyle{fancy}
%\vspace*{15pt}

\section*{Overview}

In this assignment, you'll practice some data manipulation in pandas, and perform a simple regression task using the Abalone data set from the UCI repository

Abalone are a type of mollusk (you may have eaten one before).  Each row of this data set is an individual abalone, with a variety of measurements.  You can learn about the age of an abalone by counting rings in its shell (kind of like counting tree rings).

https://archive.ics.uci.edu/ml/datasets/Abalone

This link describes the data.  Take a look at the attribute information so you know what each column is referring to.

Please put all your code into a single script {\tt hw2.py}, which will generate the image {\tt regression.png}.  Submit both files to canvas.

Please submit the following files to Canvas:
\begin{enumerate}
\item {\tt hw2.py} - a script with all your code, which will generate the figure
\item {\tt regression.png}
\end{enumerate}

 Additionally, {\bf please include comments in your code} to explain what you're doing (doesn't have to be detailed, but should be clear).


\section{Read the data into a DataFrame}

use pandas to read the data at this link into a DataFrame:\\ https://archive.ics.uci.edu/ml/machine-learning-databases/abalone/abalone.data

Hints:
\begin{itemize}
\item decide if {\tt pd.read\_table} or {\tt pd.read\_csv} is more appropriate.
\item you can pass the url provided as a string (like in the Lecture 5 exercise)
\item You'll want to pass in at least 2 keyword arguments:
\begin{itemize}
\item {\tt names} - a list of column names ({\tt `sex', `length', `diam', `height', `wt\_whole', `wt\_shucked', `wt\_viscera', `wt\_shell', `rings'})
\item {\tt index\_col=False} - indicates that pandas should just create an index for each entry (we don't have ids for the data)
\end{itemize}
\end{itemize}

\section{Set up the data for a regression problem}

The problem we're going to try to solve is to predict the number of rings in an abalone shell (the {\tt `rings'} column) from the other features. This is a proxy for the age of the animal.

This means we want a response vector $y$ that contains the data in the `rings' column, and a design matrix $X$ that contains all the data we'll use to predict the response (the other columns).

Use the {\tt pasty} library's {\tt dmatrices} function to form your data and response matrices (see lecture 6 for an example).

Hints:
\begin{itemize}
\item Since `sex' is categorical you'll want to use {\tt `C(sex)'} in your model specification.
\end{itemize}

you'll see that {\tt X} has a column called `Intercept'.  We will not need this column, so remove it from the dataframe:

{\tt X.drop(X[["Intercept"]], axis=1, inplace=True)}

\section{Split the data into train and test sets}

Now, we're going to start using Scikit learn.

Split your data into train and test sets (See lecture 6 for an example)

Set your test size to be 30\% of the data

\section{Fit a Linear Regression model to the data}

\begin{enumerate}
\item Use Scikit learn's linear regression class to fit the model:

{\tt from sklearn.linear\_model import LinearRegression}
\item Use your training data to fit the model
\item Predict the number of rings in your test data using the {\tt predict} method.
\item create a scatter plot of your prediction vs. the true number of rings.  Save this figure as {\tt regression.png} and sumbit it with your homework
\end{enumerate}

\section{(Bonus) Try another regression classifier}

Pick another regression classifier (e.g., try ridge, lasso, decision trees, nearest neighbors, ...) and repeat parts 3 and 4.  If you do this, name your image after the classifier you used e.g., {\tt lasso.png}



\end{document}
