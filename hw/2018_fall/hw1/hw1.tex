\documentclass{article}
\newcommand{\assgnnum}{1}
\newcommand{\duedate}{October 16, 10:30 AM}

\usepackage{amsmath}
%\usepackage{fullpage}
\usepackage{amssymb}
%\usepackage{bbm}
\usepackage{fancyhdr}
%\usepackage{paralist}
\usepackage{graphicx}
\usepackage{caption}
\usepackage{subcaption}
\usepackage[pdftex,colorlinks=true, urlcolor = blue]{hyperref}
\usepackage{listings}
\usepackage{color}
\usepackage{xcolor}

\definecolor{thegreen}{rgb}{0,.5,0}
\definecolor{comment-green}{rgb}{0,.3,0}
\definecolor{theblue}{rgb}{0,0,.8}
\definecolor{light-gray}{gray}{0.98}
\definecolor{comment-color}{rgb}{0,0,.8}
\definecolor{string-color}{rgb}{0,.75,0}
\definecolor{border-blue}{rgb}{0,0,.6}

\lstset{% use our version of highlighting
  language=python,      % using python
  keywordstyle={\color{teal}\bfseries},          % keywords
  commentstyle=\color{comment-color},         % comments
  stringstyle=\color{string-color},                   %strings
}

\lstset{
  basicstyle={\ttfamily\normalsize},  % use font and smaller size
  basewidth={0.5em,0.5em},
  showstringspaces=false,                   % do not emphasize spaces in strings
  tabsize=2,                                % number of spaces of a TAB
  aboveskip={0\baselineskip},               % a bit of space above
  columns=fixed,                            % nice spacing
}


\oddsidemargin 0in \evensidemargin 0in
\topmargin -0.5in \headheight 0.25in \headsep 0.25in
\textwidth 6.5in \textheight 9in
\parskip 6pt \parindent 0in \footskip 20pt

% set the header up
\fancyhead{}
\fancyhead[L]{CME193: Homework \assgnnum}
\fancyhead[R]{Due: \duedate}
%%%%%%%%%%%%%%%%%%%%%%%%%%
\renewcommand\headrulewidth{0.4pt}
\setlength\headheight{15pt}

\newcommand{\p}{\ensuremath{\mathbf{P}}}
\renewcommand{\Pr}[1]{\ensuremath{\p \left \{ #1 \right \}}}
\newcommand{\nti}{\ensuremath{n \to \infty}}
\newcommand{\I}{\ensuremath{\operatorname{I}}}
\newcommand{\One}[1]{\ensuremath{\mathbbm{1}_{\left \{ #1 \right \}}}}
\newcommand{\E}{\ensuremath{\mathbf{E}}}
\newcommand{\Ex}[2][]{\ensuremath{\E_{#1} \left[ #2 \right]}}
\newcommand{\var}{\ensuremath{\operatorname{Var}}}
\newcommand{\cov}{\ensuremath{\operatorname{Cov}}}
\newcommand{\F}{\ensuremath{\mathcal{F}}}
\newcommand{\R}{\ensuremath{\mathbb{R}}}
\newcommand{\C}{\ensuremath{\mathbb{C}}}
\newcommand{\NormRV}[2]{\ensuremath{\operatorname{N}\left(#1, #2\right)}}
\newcommand{\BetaRV}[2]{\ensuremath{\operatorname{Beta}\left(#1, #2\right)}}
\newcommand{\argmax}{\operatornamewithlimits{argmax}}
\newcommand{\x}{\mathbf{x}}
\newcommand{\A}{\mathbf{A}}
\newcommand{\bb}{\mathbf{b}}

\newcounter{points}
\setcounter{points}{0}
\newcounter{bonuspoints}
\setcounter{bonuspoints}{0}

\newcommand\setpoints[1]{\addtocounter{points}{#1}(#1 points)}
\newcommand\setpoint{\addtocounter{points}{1}(1 point)}
\newcommand\setbonuspoints[1]{\addtocounter{bonuspoints}{#1}(#1 bonus points)}
\newcommand\setbonuspoint{\addtocounter{bonuspoints}{1}(1 bonus point)}

\newcommand\printpoints{Total number of points: \value{\thepoints}}

\newcommand{\eqD}{\ensuremath{\overset{\mathcal{D}}{=}}}

\setlength{\parindent}{0in}

\begin{document}

\pagestyle{fancy}
%\vspace*{15pt}

\section*{Overview}

In this assignment, you'll read sparse matrices from files, implement a class, and implement a simple spectral embedding algorithm.

All your code should be put in a script {\tt hw1.py} which will be submitted on Canvas.

Please submit the following files to Canvas:
\begin{enumerate}
\item {\tt hw1.py} - a script with all your code, which will generate the figure
\item {\tt sbm.png}
\end{enumerate}

{\bf Please use the function, class and file names specified.}  It will be easier to read your code if everyone uses the same conventions.  Additionally, {\bf please include comments in your code} to explain what you're doing (doesn't have to be detailed, but should be clear). 


\section{Review of Graph Terminology}

A graph $G(V,E)$ is a set of vertices $V$ and edges $E\subset V\times V$.  We'll denote the number of vertices in a graph by $n$, and assign each vertex a number in $0,1,\dots, n-1$.

The adjacency matrix of a graph is a $n\times n$ matrix $A$, where
$$A_{i,j} = \begin{cases}
1 & (i,j) \in E\\
0 & (i,j) \notin E
\end{cases}$$


\section{Read A Sparse Matrix From a File}

{\tt sbm.csv} contains the adjacency matrix of a graph in the following format: each row of the file contains the contents for a single non-zero of the adjacency matrix:

{\tt
row (int), column (int), value (float)
}


Write a function that will take a file name as input and return a {\tt scipy.sparse.coo\_matrix}

{\bf call this function {\tt read\_coo} }

If you prefer to work with another sparse matrix type, you can always convert once you have created a COO matrix.

\section{Create A Sparse + Rank-1 Matrix Class}

In the next part, it will be convenient to have a class to represent matrices of the form
$$ S + \alpha uv^T$$
where $S$ is a sparse matrix, $u, v$ are vectors, and $\alpha$ is a scalar.  This allows us to use the structure of the matrix to avoid forming a dense array.

{\bf use the class name {\tt sparse\_rank1} } 

Write a class definition that
\begin{itemize}
\item initializes an object given a sparse matrix $S$, numpy arrays $u$ and $v$ and a float $\alpha$ (store each of these inputs).  Give the object another field {\tt shape}, which is set to be the same as {\tt S.shape}
\item implements a {\tt dot} method, which performs matrix-vector multiplication
\end{itemize}

\section{Power Method}
In lecture 2, you did an exercise in which you implemented power method for a matrix.  Recall, this function finds the eigenpair $(\lambda, v)$ with largest $\lambda$ such that $Av = \lambda v$ for a matrix $A$.  Note that the implementation only required that $A$ have a {\tt dot} method that performed matrix-vector multiplication, so you can use it with your new matrix class.

Import this function to your script, and modify it if needed to work with your {\tt sparse\_rank1} matrix class.

{\bf call this function {\tt power\_method} }

\section{Spectral Embedding}

{\tt sbm.csv} contains the adjacency matrix of a graph generated using the stochastic block model.  Load this into a sparse matrix {\tt A}.

A spectral embedding assigns vertices of a graph coordinates in Euclidean space that can be used to visualize the graph.  One way to generate a spectral embedding is to calculate the top $k$ vectors of the adjacency matrix, and embed in $\mathbb{R}^k$.  We'll do this for $k=2$.

Find the top two eigenvectors of adjacency matrix using the power method using the following deflation algorithm:
\begin{enumerate}
\item Calculate the top eigenpair $(\lambda_1, v_1)$ of $A$
\item Calculate the top eigenpair $(\lambda_2, v_2)$ of $(A - \lambda_1 v_1 v_1^T)$
\end{enumerate}
You can either wrap this in a function, or just execute it directly in the script.

Use PyPlot to generate a scatter plot of $v_1$ vs $v_2$.  {\bf Save this plot as {\tt sbm.png} and submit it on Canvas.}

Note that this is a very simplified version of a specific spectral clustering algorithm.  If you want to know more about this spectral clustering setup, I recommend the paper:

``Robust and efficient multi-way spectral clustering'' by A. Damle, V. Minden, and L. Ying. (2017) \\
{\tt https://arxiv.org/abs/1609.08251}




\section*{Hints}

You don't need to use the hints to complete the assignment - it's ok if you want to use functions other than the ones mentioned.

\begin{itemize}
\item Reading a file: check out {\tt numpy.readtxt}
\item To specify a data type in a numpy array, pass in a type e.g. {\tt np.array(v, int)}
\item $ (S + \alpha uv^T) x = Sx + \alpha u (v^T x)$
\item Save a figure: check out {\tt savefig} in pyplot
\item You should see 2 clusters when you generate the figure
\end{itemize}





\end{document}
